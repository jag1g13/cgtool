\documentclass[10pt]{article}

%\usepackage[T1]{fontenc}
\usepackage{url}

\begin{document}
\title{CGTOOLS}
\author{James Graham, University of Southampton - J.A.Graham@soton.ac.uk}

\maketitle

\tableofcontents

\pagebreak

\section{Included Programs}
CGTOOLS currently contains three programs: CGTOOL - semi-automated coarse grain mapping, RAMSi - analysis of membrane simulations, and xtc-length - brief details about a Gromacs \path{xtc} file.

\subsection{CGTOOL}
CGTOOL is a tool to aid in the creation and analysis of coarse-grained (CG) molecular dynamics (MD) models.  It takes simulated trajectories from all-atom or united-atom (herein referred to as AA) simulations using the popular GROMACS simulation package and generates the necessary input files for a CG simulation using a user defined mapping.

\subsection{RAMSi}
RAMSi stands for Rapid Analysis of Membrane Simulations and is a tool which performs several common analyses of biomembranes.  Included are bilayer thickness and lipid surface area, with experimental support for membrane local curvature.

\subsection{xtc-length}
The program xtc-length reads a Gromacs \path{xtc} file and returns the number of frames, time in nanoseconds and the number of atoms.  It is significantly faster that the Gromacs tool \verb|gmx check| as it does not perform any error checking on the \path{xtc} file and reads only frame headers.

\6section{Basic Use}
The first CG model used with CGTOOL was the MARTINI forcefield and this remains the easiest to use.  This is the recommended use for those new to CG models.

\subsection{Input Files}
The program requires files in GROMACS XTC and GRO format which are the trajectory of an AA simulation.  These may be generated by another simulation package as long as they can be converted into XTC and GRO format.

\subsubsection{The Configuration File}


\section{Advanced Use}

\section{Future Improvements}
There are many improvements planned for future releases, of which some of the most important are:
\begin{itemize}
\item Automatic selection of MARTINI bead type based on charge and dipole moment
\item More complete calculation of electrostatics for CG models including multipoles
\item Support for simulation packages other than GROMACS
\item Rigorous Lennard-Jones parameter calculations for models requiring them
\end{itemize}


\end{document}