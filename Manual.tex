\documentclass[10pt]{article}

\usepackage[T1]{fontenc}

\begin{document}
\title{CGTOOL}
\author{James Graham - J.A.Graham@soton.ac.uk}

\maketitle

\tableofcontents

\pagebreak

\section{About CGTOOL}
CGTOOL is being developed to aid in the creation and analysis of coarse-grained (CG) molecular dynamics (MD) models.  It is capable of taking simulated trajectories from all-atom or united-atom (both now referred to as AA) simulations using the popular GROMACS simulation package and generating the necessary input files for a CG simulation using a user defined mapping.

It is written in C++ and compiled using CMake for portability and speed.

\section{Basic Use}
The first CG model used with CGTOOL was the MARTINI forcefield and this remains the easiest to use.  This is the recommended use for anyone new to CG models.

\subsection{Input Files}
The program requires files in GROMACS XTC and GRO format which are the trajectory of an AA simulation.  These may be generated by another simulation package as long as they can be converted into XTC and GRO format.

\subsubsection{The Configuration File}


\section{Advanced Use}

\section{Future Improvements}
There are many improvements planned for future releases, of which some of the most important are:
\begin{itemize}
\item Automatic selection of MARTINI bead type based on charge and dipole moment
\item More complete calculation of electrostatics for CG models including multipoles
\item Support for simulation packages other than GROMACS
\item Rigorous Lennard-Jones parameter calculations for models requiring them
\end{itemize}


\end{document}